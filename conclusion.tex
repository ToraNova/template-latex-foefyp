%!TEX ROOT = thesis.tex
\chapter{Conclusion}
This chapter describes briefly and concisely the overall achievement of the project in terms of what have been done, what are the features, what are the functions, etc..

Notes: You may write your conclusion in several paragraphs. Note that conclusions are written in a case by case basis. Hence, this typical format is used as a guide or reference for you to write conclusions. First and second person pronouns (I, we, you, me, my, etc.) should be minimized or avoided.

Conclusion CANNOT include the following items: 1. Issues related to personal, e.g. learned a lot of things from this project. 2. Any issues or works that are not produced from your project (except comparison cases with another person’s work). 3. Any issues that are not discussed in discussion chapter.

Individual conclusions: These individual conclusions are made based on the chapter ‘discussion of findings’. Each discussion in the discussion chapter is concluded here without further discussion. In some cases, a conclusion can be made based on several discussions. Conclusions are made in terms of advantages, disadvantages, limitations, dependencies, affecting factors, problems, etc. All the conclusions should be in justified or confirmed (either good or bad) manner and should not look like discussion.

Overall conclusion: In some cases, an overall conclusion can be made based on the individual conclusions which can be combined into one.
